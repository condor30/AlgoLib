\addcontentsline{toc}{section}{СПИСОК ИСПОЛЬЗОВАННЫХ ИСТОЧНИКОВ}

\begin{thebibliography}{99}
	
	\bibitem{freeman} Фримен, А. Практикум по программированию на Python / А. Фримен. – Москва : Вильямс, 2015. – 720 с. – ISBN 978-5-8459-1799-7.
	
	\bibitem{stroustrup} Страуструп, Б. Язык программирования C++. Лекции и упражнения / Б. Страуструп. – Санкт-Петербург : БХВ-Петербург, 2018. – 880 с. – ISBN 978-5-9579-2180-5.
	
	\bibitem{goodman} Гудман, Д. Полное руководство по разработке веб-приложений с использованием Django / Д. Гудман. – Москва : ДМК Пресс, 2017. – 624 с. – ISBN 978-5-94074-625-8.
	
	\bibitem{vanderplas} Вандер Плас, Дж. Python для сложных задач. Наука о данных и машинное обучение / Дж. Вандер Плас. – Санкт-Петербург : Питер, 2016. – 560 с. – ISBN 978-5-496-01049-8.
	
	\bibitem{lutz} Лутц, М. Изучаем Python. Программирование игр, визуализация данных, веб-приложения / М. Лутц. – Москва : ДМК Пресс, 2019. – 864 с. – ISBN 978-5-97060-729-5.
	
	\bibitem{tanenbaum} Таненбаум, Э. Архитектура компьютера / Э. Таненбаум. – Санкт-Петербург : Питер, 2014. – 896 с. – ISBN 978-5-496-01049-8.
	
	\bibitem{stallings} Сталлингс, У. Компьютерные сети. Принципы, технологии, протоколы / У. Сталлингс. – Москва : Издательский дом Вильямс, 2017. – 864 с. – ISBN 978-5-8459-2125-3.
	
	\bibitem{martin} Мартин, Р. Чистый код. Создание, анализ и рефакторинг / Р. Мартин. – Санкт-Петербург : Питер, 2018. – 464 с. – ISBN 978-5-4461-1089-3.
	
	\bibitem{kalender} Календер, Д. Принципы объектно-ориентированного программирования на С++ с примерами на C# и Java / Д. Календер. – Москва : ДМК Пресс, 2016. – 512 с. – ISBN 978-5-97060-858-2.
	
	\bibitem{hunter} Хантер, Э. Программирование на Python 3. Подробное руководство / Э. Хантер. – Москва : ДМК Пресс, 2019. – 560 с. – ISBN 978-5-97060-827-8.
	
\end{thebibliography}
