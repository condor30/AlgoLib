\section*{ВВЕДЕНИЕ}
\addcontentsline{toc}{section}{ВВЕДЕНИЕ}
Библиотека алгоритмов - это набор предварительно написанных функций, классов или модулей, которые обеспечивают эффективные и готовые к использованию реализации обычно используемых алгоритмов.

Алгоритмические библиотеки сыграли решающую роль в развитии вычислительной техники и решении проблем. Первоначально появившиеся в 1950-х и 1960-х годах, эти библиотеки были разработаны для предоставления эффективных решений алгоритмических задач, возникавших при обработке данных на ранних компьютерах. Со временем эти библиотеки эволюционировали и теперь включают в себя широкий спектр алгоритмов, от классических алгоритмов сортировки до алгоритмов расширенного поиска, методов оптимизации и алгоритмов построения графиков.

В настоящее время существует множество графических редакторов. Они имеют различную стоимость и набор функций. Некоторые из них предназначены для профессиональной работы с изображениями, а другие - для обычных пользователей.

\emph{Цель данной работы} - разработка библиотеки алгоритмов. Для достижения этой цели необходимо решить \emph{следующие задачи:}
\begin{itemize}
	\item Провести анализ предметной области;
	\item Разработать концептуальную модель библиотеки алгоритмов;
	\item Реализовать библиотеку.
\end{itemize}

\emph{Структура и объем работы.} Отчет состоит из введения, 4 основных разделов, заключения, списка использованных источников и 2 приложений. Текст отчета о квалификации составляет \formbytotal{page}{страниц}{е}{ам}{ам}.

\emph{Во введении} сформулирована цель работы, определены задачи разработки, описана структура работы и представлен краткий обзор каждого из разделов.

\emph{В первом разделе} на этапе описания технических характеристик предметной области собирается информация о существующих библиотеках алгоритмов.

\emph{Во втором разделе} на этапе составления технического задания устанавливаются требования к разрабатываемой библиотеке алгоритмов.

\emph{В третьем разделе} на этапе технического проектирования представлены проектные решения для библиотеки алгоритмов.

\emph{В четвертом разделе} представлен список классов и их методов, использованных при разработке библиотеки алгоритмов, а также проведено тестирование разработанной библиотеки.

В заключении изложены основные результаты работы, полученные в процессе разработки.

В приложении А представлен графический материал.
В приложении Б представлены фрагменты исходного кода.





